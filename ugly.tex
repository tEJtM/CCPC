\documentclass[11pt]{article}
\usepackage{fullpage}
\usepackage{color}
\usepackage{graphicx}
\usepackage{epsfig}
\usepackage{amsthm}
\usepackage{latexsym}
\usepackage{amssymb}
\usepackage{amsmath}
\usepackage{parcolumns}
\usepackage{ulem}
\newcommand{\secdes}{\subsection{Description}}
\newcommand{\secinput}{\subsection{Input}}
\newcommand{\secoutput}{\subsection{Output}}
\newcommand{\secsample}{\subsection{Sample}}
\newcommand{\sechint}{\subsection{Hint}}
\newcommand{\sample}[2]{\begin{tabular}{@{\extracolsep{\fill}}|l|l|}
\hline
Sample Input & Sample Output \\
\hline
\begin{tabular}{@{}l@{}}#1\end{tabular} & \begin{tabular}{l}#2\end{tabular}\\
\hline
\end{tabular}}%
\begin{document}
\section{Ugly problem}

\secdes

Everyone hate ugly problems.

You are given a positive integer. You must represent that number by sum of palindromic numbers.

A palindromic number is a positive integer such that if you write out that integer as a string in decimal  without leading zeroes, the string is an palindrome. For examples, $1$ is a palindromic number and $10$ is not.

\secinput

\indent The input contains multiple test cases.

The first line contains an integer $T$, the number of test cases.

Then $T$ lines follows. In each line, there's a positive integer $s(1\le s\le 10^{1000})$.

\secoutput
For each test case, output ``Case \#XXX:'' on the first line where XXX is the number of that test case starting from $1$.
Then output the number of palindromic numbers you used, $n$, on one line. $n$ must be no more than $50$. 
Then output $n$ lines, each containing one of your palindromic numbers. Their sum must be exactly $s$.

\secsample
\sample{
2\\
18\\
1000000000000
} {
Case \#1:\\
2\\
9\\
9\\
Case \#2:\\
2\\
999999999999\\
1
}
\sechint
$9+9=18$

$999999999999+1=1000000000000$
\end{document}
